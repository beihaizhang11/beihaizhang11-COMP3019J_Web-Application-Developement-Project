\subsubsection{Resource Management}

\paragraph{Description}
Resource Management is a critical use case that enables librarians to oversee and maintain the university library's collection of materials. This functionality allows librarians to add new books, journals, and other resources to the system, edit existing resource information, and remove outdated or damaged items from the catalog. Librarians can update details such as titles, authors, publication dates, ISBN numbers, and physical locations within the library. This use case is essential for keeping the library's digital catalog accurate and up-to-date, facilitating efficient resource discovery and access for all library users.

\paragraph{Actors}
\begin{itemize}
    \item Librarian
\end{itemize}

\paragraph{Triggers}
\begin{itemize}
    \item Acquisition of new library resources
    \item Need to update information for existing resources
    \item Removal of outdated or damaged materials
    \item Regular catalog maintenance and auditing
\end{itemize}

\paragraph{Preconditions}
\begin{itemize}
    \item Librarian is logged into their account
    \item Librarian is on the Resource Management section of the library system
    \item Librarian has necessary permissions to modify the resource catalog
\end{itemize}

\paragraph{Postconditions}
\begin{itemize}
    \item Resource information is successfully added, updated, or removed from the system
    \item Changes are immediately reflected in the library catalog
    \item System logs the resource management activities
    \item Users can see the updated resource information when searching the library catalog
\end{itemize}

\paragraph{Data Outcomes}
\begin{itemize}
    \item \textbf{CREATE} - New resource entries in the catalog
    \item \textbf{READ} - Existing resource information
    \item \textbf{UPDATE} - Resource details (e.g., title, author, location, availability status)
    \item \textbf{DELETE} - Entries for resources no longer in the library collection
\end{itemize}

\subsubsection{User Borrowing Records Management}

\paragraph{Description}
User Borrowing Records Management is a crucial use case that allows librarians to oversee and manage the borrowing activities of library users. This functionality enables librarians to record new loans, process returns, manage renewals, and maintain a comprehensive history of user borrowing activities. Librarians can use this system to track due dates, issue reminders for overdue items, and handle fines or penalties. This use case is essential for maintaining an organized and efficient library system, ensuring proper circulation of resources, and providing accountability for borrowed materials.

\paragraph{Actors}
\begin{itemize}
    \item Librarian
\end{itemize}

\paragraph{Triggers}
\begin{itemize}
    \item User borrows a library resource
    \item User returns a borrowed item
    \item Request for loan renewal
    \item Need to check a user's borrowing history
    \item Overdue item management
    \item Fine or penalty assessment and collection
\end{itemize}

\paragraph{Preconditions}
\begin{itemize}
    \item Librarian is logged into their account
    \item Librarian is on the User Borrowing Records section of the library system
    \item Librarian has necessary permissions to access and modify borrowing records
\end{itemize}

\paragraph{Postconditions}
\begin{itemize}
    \item Borrowing activity is accurately recorded in the system
    \item User's borrowing history is updated
    \item Due dates and loan statuses are correctly reflected
    \item Fines or penalties, if any, are properly assessed and recorded
    \item System logs the borrowing record management activities
\end{itemize}

\paragraph{Data Outcomes}
\begin{itemize}
    \item \textbf{CREATE} - New borrowing record entries
    \item \textbf{READ} - Existing borrowing histories, due dates, fine records
    \item \textbf{UPDATE} - Loan statuses, renewal information, fine payments
    \item \textbf{DELETE} - Resolved overdue records or paid fines (if applicable)
\end{itemize}