\subsubsection{E-bike Management}

\paragraph{Description}
E-bike Management is a specialized use case that enables security personnel to oversee and regulate the use of electric bicycles on campus. This functionality allows security staff to review and manage E-bike registration information submitted by students, approve or reject registration requests, monitor E-bike usage patterns, and ensure compliance with campus policies. The system helps maintain safety standards, manage parking allocations, and track E-bikes in case of theft or accidents.

\paragraph{Actors}
\begin{itemize}
    \item Security Personnel
\end{itemize}

\paragraph{Triggers}
\begin{itemize}
    \item New E-bike registration submission
    \item Need to review existing E-bike registrations
    \item Reported E-bike-related incidents or violations
    \item Regular E-bike usage audits
\end{itemize}

\paragraph{Preconditions}
\begin{itemize}
    \item Security personnel is logged into their account
    \item Security personnel is on the E-bike Management section of their dashboard
\end{itemize}

\paragraph{Postconditions}
\begin{itemize}
    \item E-bike registration requests are processed (approved or rejected)
    \item E-bike database is updated with the latest information
    \item Incident reports or policy violations are recorded and addressed
    \item System logs all E-bike management activities
\end{itemize}

\paragraph{Data Outcomes}
\begin{itemize}
    \item \textbf{READ} - E-bike registration information, usage data
    \item \textbf{UPDATE} - Registration status, incident reports
    \item \textbf{CREATE} - New policy violation records, audit logs
    \item \textbf{DELETE} - Outdated or invalid E-bike registrations
\end{itemize}

