\subsubsection{Course Register}

\paragraph{Description}
Course Management for students is a fundamental use case that enables students to interact with their academic curriculum through the university campus management system. This functionality allows students to view available courses, register for classes during designated periods, and drop courses when necessary. The system enforces registration rules, such as prerequisites and credit limits, ensuring students can only enroll in appropriate courses. Additionally, students can access their current course schedule, view course details, and receive notifications about important course-related events or deadlines.

\paragraph{Actors}
\begin{itemize}
    \item Student
\end{itemize}

\paragraph{Triggers}
\begin{itemize}
    \item Start of course registration period
    \item Need to adjust course schedule
    \item Approaching add/drop deadline
    \item Viewing current or future course offerings
\end{itemize}

\paragraph{Preconditions}
\begin{itemize}
    \item Student is logged into their account
    \item Student is on the Course Management section of their dashboard
    \item Course registration period is active (for registration actions)
\end{itemize}

\paragraph{Postconditions}
\begin{itemize}
    \item Student's course schedule is updated with new registrations or drops
    \item System records all course registration activities
    \item Student receives confirmation of successful course actions
    \item Course enrollment numbers are updated in the system
\end{itemize}

\paragraph{Data Outcomes}
\begin{itemize}
    \item \textbf{READ} - Available courses, current course schedule, course details
    \item \textbf{CREATE} - New course registrations
    \item \textbf{UPDATE} - Student's course schedule
    \item \textbf{DELETE} - Dropped courses from student's schedule
\end{itemize}

\subsubsection{Grade Inquiry}

\paragraph{Description}
Grade Inquiry is an essential use case that allows students to access and review their academic performance within the university campus management system. This functionality enables students to view their current grades for enrolled courses, as well as a comprehensive transcript of their past academic achievements. Students can track their progress throughout the semester, view final grades after course completion, and monitor their overall GPA. The system provides a secure and confidential method for students to access this sensitive information, ensuring that only authorized individuals can view a student's academic records.

\paragraph{Actors}
\begin{itemize}
    \item Student
\end{itemize}

\paragraph{Triggers}
\begin{itemize}
    \item Student wishes to check current course grades
    \item End of academic term when final grades are released
    \item Need to review academic history for program planning
    \item Preparation for academic advising sessions
\end{itemize}

\paragraph{Preconditions}
\begin{itemize}
    \item Student is logged into their account
    \item Student is on the Grade Inquiry section of their dashboard
    \item Grades have been entered and released by instructors
\end{itemize}

\paragraph{Postconditions}
\begin{itemize}
    \item Student successfully views their requested grade information
    \item System logs the grade inquiry access for security purposes
    \item No changes are made to the grade data (read-only access)
\end{itemize}

\paragraph{Data Outcomes}
\begin{itemize}
    \item \textbf{READ} - Current course grades, historical grades, overall GPA, academic transcript
    \item \textbf{CREATE} - Log entry of grade inquiry access
\end{itemize}

\subsubsection{E-bike Registration}

\paragraph{Description}
E-bike Registration is a specialized use case that allows students to register their electric bicycles with the university's system. This functionality enables students to input and store information about their e-bikes, including brand, model, license plate number, and potentially upload images of the e-bike and relevant documents. The registration process helps the university maintain a record of e-bikes on campus, enhancing security measures and facilitating proper management of parking and charging facilities. It also assists in the identification of e-bikes in case of theft or accidents.

\paragraph{Actors}
\begin{itemize}
    \item Student
\end{itemize}

\paragraph{Triggers}
\begin{itemize}
    \item Student acquires a new e-bike
    \item University policy requires e-bike registration
    \item Student needs to update existing e-bike information
\end{itemize}

\paragraph{Preconditions}
\begin{itemize}
    \item Student is logged into their account
    \item Student is on the E-bike Registration section of their dashboard
    \item Student has all necessary e-bike information and documents ready
\end{itemize}

\paragraph{Postconditions}
\begin{itemize}
    \item E-bike information is successfully registered or updated in the system
    \item Student receives a confirmation of successful registration
    \item Security department is notified of the new e-bike registration
\end{itemize}

\paragraph{Data Outcomes}
\begin{itemize}
    \item \textbf{CREATE} - New e-bike registration entry
    \item \textbf{UPDATE} - Existing e-bike information (if applicable)
    \item \textbf{READ} - Current e-bike registration details
\end{itemize}