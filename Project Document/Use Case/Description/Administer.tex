\subsection{Administrater Functions}
\subsubsection{User Management}

\paragraph{Description}
User Management is a critical function performed by the system administrator to maintain and control user accounts within the university campus management system. This use case encompasses the creation of new accounts for students, teachers, librarians, and security personnel, as well as the modification and deletion of existing accounts. The administrator can update user information, reset passwords, and adjust access permissions as needed. Additionally, this use case includes the ability to view and audit user account details, ensuring the system's security and integrity.

\paragraph{Actors}
\begin{itemize}
    \item System Administrator
\end{itemize}

\paragraph{Triggers}
\begin{itemize}
    \item New user needs to be added to the system
    \item Existing user information requires updating
    \item User account needs to be removed from the system
    \item Regular account audit and maintenance schedules
\end{itemize}

\paragraph{Preconditions}
\begin{itemize}
    \item Administrator is logged into the system with appropriate credentials
    \item Administrator is on the User Management section of the Admin Dashboard
\end{itemize}

\paragraph{Postconditions}
\begin{itemize}
    \item User accounts are successfully created, updated, or deleted as required
    \item Changes in user accounts are reflected immediately in the system
    \item System logs record all significant account management actions
\end{itemize}

\paragraph{Data Outcomes}
\begin{itemize}
    \item \textbf{CREATE} - New user accounts
    \item \textbf{READ} - Existing user account information
    \item \textbf{UPDATE} - User account details, permissions
    \item \textbf{DELETE} - User accounts no longer needed
\end{itemize}

\subsubsection{System Configuration}

\paragraph{Description}
System Configuration is a crucial use case that allows the system administrator to manage and customize various aspects of the university campus management system. This functionality encompasses setting up parameters for course management, configuring resource-related settings, establishing activity rules, and adjusting user interface options. The administrator can also define security policies, schedule system maintenance, and implement new university-wide regulations through this interface. By providing centralized control over these settings, the System Configuration use case ensures the system remains flexible, secure, and aligned with the institution's evolving needs.

\paragraph{Actors}
\begin{itemize}
    \item System Administrator
\end{itemize}

\paragraph{Triggers}
\begin{itemize}
    \item Need to modify system behavior or policies
    \item Start of a new academic year or semester
    \item Implementation of new university policies
    \item System performance optimization requirements
    \item Security update necessities
\end{itemize}

\paragraph{Preconditions}
\begin{itemize}
    \item Administrator is logged into the system with appropriate credentials
    \item Administrator is on the System Configuration section of the Admin Dashboard
\end{itemize}

\paragraph{Postconditions}
\begin{itemize}
    \item System configurations are successfully updated
    \item New settings are applied system-wide or to specified user groups
    \item All configuration changes are logged for auditing purposes
    \item Users are notified of relevant system changes if necessary
\end{itemize}

\paragraph{Data Outcomes}
\begin{itemize}
    \item \textbf{READ} - Current system configuration settings
    \item \textbf{UPDATE} - System configuration parameters
    \item \textbf{CREATE} - New configuration options (if the system allows for custom configurations)
    \item \textbf{DELETE} - Obsolete configuration options
\end{itemize}